%% Generated by Sphinx.
\def\sphinxdocclass{report}
\documentclass[a4paper,10pt,english,openany,oneside]{sphinxmanual}
\ifdefined\pdfpxdimen
   \let\sphinxpxdimen\pdfpxdimen\else\newdimen\sphinxpxdimen
\fi \sphinxpxdimen=.75bp\relax

\PassOptionsToPackage{warn}{textcomp}
\usepackage[utf8]{inputenc}
\ifdefined\DeclareUnicodeCharacter
% support both utf8 and utf8x syntaxes
  \ifdefined\DeclareUnicodeCharacterAsOptional
    \def\sphinxDUC#1{\DeclareUnicodeCharacter{"#1}}
  \else
    \let\sphinxDUC\DeclareUnicodeCharacter
  \fi
  \sphinxDUC{00A0}{\nobreakspace}
  \sphinxDUC{2500}{\sphinxunichar{2500}}
  \sphinxDUC{2502}{\sphinxunichar{2502}}
  \sphinxDUC{2514}{\sphinxunichar{2514}}
  \sphinxDUC{251C}{\sphinxunichar{251C}}
  \sphinxDUC{2572}{\textbackslash}
\fi
\usepackage{cmap}
\usepackage[T1]{fontenc}
\usepackage{amsmath,amssymb,amstext}
\usepackage{babel}


\usepackage{amsmath,amsfonts,amssymb,amsthm}

\usepackage{fncychap}
\usepackage{sphinx}
\sphinxsetup{hmargin={0.7in,0.7in}, vmargin={1in,1in},         verbatimwithframe=true,         TitleColor={rgb}{0,0,0},         HeaderFamily=\rmfamily\bfseries,         InnerLinkColor={rgb}{0,0,0},         OuterLinkColor={rgb}{0,0,0}}
\fvset{fontsize=\small}
\usepackage{geometry}


% Include hyperref last.
\usepackage{hyperref}
% Fix anchor placement for figures with captions.
\usepackage{hypcap}% it must be loaded after hyperref.
% Set up styles of URL: it should be placed after hyperref.
\urlstyle{same}
\addto\captionsenglish{\renewcommand{\contentsname}{Contents:}}

\usepackage{sphinxmessages}
\setcounter{tocdepth}{1}


        %%%%%%%%%%%%%%%%%%%% Meher %%%%%%%%%%%%%%%%%%
        %%%add number to subsubsection 2=subsection, 3=subsubsection
        %%% below subsubsection is not good idea.
        \setcounter{secnumdepth}{3}
        %
        %%%% Table of content upto 2=subsection, 3=subsubsection
        \setcounter{tocdepth}{2}

        \usepackage{amsmath,amsfonts,amssymb,amsthm}
        \usepackage{graphicx}

        %%% reduce spaces for Table of contents, figures and tables
        %%% it is used "\addtocontents{toc}{\vskip -1.2cm}" etc. in the document
        \usepackage[notlot,nottoc,notlof]{}

        \usepackage{color}
        \usepackage{transparent}
        \usepackage{eso-pic}
        \usepackage{lipsum}

        \usepackage{footnotebackref} %%link at the footnote to go to the place of footnote in the text

        %% spacing between line
        \usepackage{setspace}
        %%%%\onehalfspacing
        %%%%\doublespacing
        \singlespacing


        %%%%%%%%%%% datetime
        \usepackage{datetime}

        \newdateformat{MonthYearFormat}{%
            \monthname[\THEMONTH], \THEYEAR}


        %% RO, LE will not work for 'oneside' layout.
        %% Change oneside to twoside in document class
        \usepackage{fancyhdr}
        \pagestyle{fancy}
        \fancyhf{}

        %%% Alternating Header for oneside
        \fancyhead[L]{\ifthenelse{\isodd{\value{page}}}{ \small \nouppercase{\leftmark} }{}}
        \fancyhead[R]{\ifthenelse{\isodd{\value{page}}}{}{ \small \nouppercase{\rightmark} }}

        %%% Alternating Header for two side
        %\fancyhead[RO]{\small \nouppercase{\rightmark}}
        %\fancyhead[LE]{\small \nouppercase{\leftmark}}

        %% for oneside: change footer at right side. If you want to use Left and right then use same as header defined above.
        \fancyfoot[R]{\ifthenelse{\isodd{\value{page}}}{{\tiny Meher Krishna Patel} }{\href{http://pythondsp.readthedocs.io/en/latest/pythondsp/toc.html}{\tiny PythonDSP}}}

        %%% Alternating Footer for two side
        %\fancyfoot[RO, RE]{\scriptsize Meher Krishna Patel (mekrip@gmail.com)}

        %%% page number
        \fancyfoot[CO, CE]{\thepage}

        \renewcommand{\headrulewidth}{0.5pt}
        \renewcommand{\footrulewidth}{0.5pt}

        \RequirePackage{tocbibind} %%% comment this to remove page number for following
        \addto\captionsenglish{\renewcommand{\contentsname}{Table of contents}}
        %\addto\captionsenglish{\renewcommand{\listfigurename}{List of figures}}
        %\addto\captionsenglish{\renewcommand{\listtablename}{List of tables}}
        \addto\captionsenglish{\renewcommand{\chaptername}{Chapter}}


        %%reduce spacing for itemize
        \usepackage{enumitem}
        \setlist{nosep}

        %%%%%%%%%%% Quote Styles at the top of chapter
        \usepackage{epigraph}
        \setlength{\epigraphwidth}{0.8\columnwidth}
        \newcommand{\chapterquote}[2]{\epigraphhead[60]{\epigraph{\textit{#1}}{\textbf {\textit{--#2}}}}}
        %%%%%%%%%%% Quote for all places except Chapter
        \newcommand{\sectionquote}[2]{{\quote{\textit{``#1''}}{\textbf {\textit{--#2}}}}}
    

\title{Share}
\date{May 16, 2020}
\release{1.0.0}
\author{Francesco Coppola, Stefano Perniola}
\newcommand{\sphinxlogo}{\sphinxincludegraphics{logo.png}\par}
\renewcommand{\releasename}{Release}
\makeindex
\begin{document}

\pagestyle{empty}

        \pagenumbering{Roman} %%% to avoid page 1 conflict with actual page 1

        \begin{titlepage}
            \centering

            \vspace*{40mm} %%% * is used to give space from top
            \textbf{\Huge {Share, a metaprogramming pattern}}

            \vspace{0mm}
            \begin{figure}[!h]
                \centering
                \includegraphics[scale=0.3]{logo.png}
            \end{figure}

            \vspace{0mm}
            \Large \textbf{{Francesco Coppola, Stefano Perniola}}

            \small Created on : March, 2020

            \vspace*{0mm}
            \small  Last updated : \MonthYearFormat\today


            %% \vfill adds at the bottom
            \vfill
            \small \textit{Project realized for the University of Camerino during the academic year 2019/2020}}
        \end{titlepage}

        \clearpage
        \pagenumbering{roman}
        \tableofcontents
        \clearpage
        \pagenumbering{arabic}

        
\pagestyle{plain}
 
\pagestyle{normal}
\phantomsection\label{\detokenize{index::doc}}

\begin{quote}

\sphinxstyleemphasis{“It’s easy to play any musical instrument: all you have to do is touch the right key at the right time and the instrument will play itself.”} %
\begin{footnote}[1]\sphinxAtStartFootnote
\sphinxhref{https://en.wikipedia.org/wiki/Johann\_Sebastian\_Bach}{Johann Sebastian Bach, german composer and musician}
%
\end{footnote}
\end{quote}

Within the following pages it will be possible to find the documentation
generated for the \sphinxstylestrong{Share} project.

The development of this code is to be attributed to the students \sphinxstylestrong{Francesco Coppola} and \sphinxstylestrong{Stefano Perniola},
while the part of abstract description and idealization of the pattern to the teacher \sphinxstylestrong{Rosario Culmone}.


\chapter{Introduction}
\label{\detokenize{introduction:introduction}}\label{\detokenize{introduction::doc}}
There are millions of home automation devices in the world and billions in the
future. Many of these interact with difficulty due to the \sphinxstylestrong{absence of a standard} %
\begin{footnote}[1]\sphinxAtStartFootnote
\sphinxstyleemphasis{It’s worth noting that the efforts of both the ISO/IEC and the IETF and IRTF have some limitations from a practical perspective. They aren’t standards as some IT pros might understand them. They are not detailed blueprints that engineers can design to.}
%
\end{footnote}. In this project, we suggest a vision of how we can overcome the problems
of interaction in pear\sphinxhyphen{}to\sphinxhyphen{}pear way.

Instead of looking for a \sphinxstylestrong{network standard},
we propose a programming standard through a design pattern. The proposed
pattern design does not travel on data network but code. This feature allows
not only to exceed the limits of data\sphinxhyphen{}oriented standards but \sphinxstylestrong{to make up for services on the fly}.

Verification mechanisms that guarantee correctness control the dynamic composition. Finally, an example is presented using the \sphinxstylestrong{Lua} %
\begin{footnote}[2]\sphinxAtStartFootnote
\sphinxstyleemphasis{Lua is a lightweight, high\sphinxhyphen{}level, multi\sphinxhyphen{}paradigm programming language designed primarily for embedded use in applications.}
%
\end{footnote}
programming language.


\chapter{Pattern}
\label{\detokenize{pattern:pattern}}\label{\detokenize{pattern::doc}}
The design pattern \sphinxstylestrong{Share} is made up of three classes:
\begin{itemize}
\item {} 
Share

\item {} 
Service

\item {} 
Feature

\end{itemize}

\sphinxstylestrong{Share} manages the common space for sharing services and services
with \sphinxstylestrong{Feature} the implementation of the service.

Each service \sphinxstylestrong{must} subscribe to at least one sharing service provider to allow others to use the
service.


\section{Description}
\label{\detokenize{pattern:description}}
\noindent\sphinxincludegraphics{{pattern}.png}

The share pattern is a \sphinxstylestrong{metaprogramming pattern} %
\begin{footnote}[1]\sphinxAtStartFootnote
\sphinxstyleemphasis{Metaprogramming is a programming technique in which computer programs have the ability to treat other programs as their data. It means that a program can be designed to read, generate, analyze or transform other programs, and even modify itself while running.}
%
\end{footnote} since some parts
of the service are known only at the time of execution and depend on the state
of the system

The definition of a service requires the coding of a function
\sphinxcode{\sphinxupquote{function}} and its \sphinxcode{\sphinxupquote{daemon}} resident component, if any. A \sphinxcode{\sphinxupquote{pre}} predicate specifies the
preconditions for function.

A string attribute identifies service in a unique world. For example an \sphinxstylestrong{SNMP MIB} can be
used to identify a service.

\begin{sphinxadmonition}{note}{Note:}
A management information base \sphinxstylestrong{(MIB)} is a database used for managing the
entities in a communication network. Most often associated with the
Simple Network Management Protocol \sphinxstylestrong{(SNMP)}, the term is also used more
generically in contexts such as in OSI/ISO Network management model.
\end{sphinxadmonition}

In \sphinxcode{\sphinxupquote{function}} coding there may be calls to external services that are specified by implementations of
\sphinxstylestrong{Feature}. The \sphinxcode{\sphinxupquote{id}} attribute defines a regular expression that describes semantically the service requested. The invocation of a service within \sphinxcode{\sphinxupquote{function}}
occurs through the call invocation relating to a specification present in features.

The \sphinxcode{\sphinxupquote{call}} operation invokes discovery with an id attribute to identify everyone
the services subscribed to \sphinxstylestrong{Share} to which the calling service is subscribed. If he comes
produced a non\sphinxhyphen{}null set of services, the next phase of invocation of services.

The primitive \sphinxcode{\sphinxupquote{get}} is used for this phase get code from a service


\section{Sequence Diagram}
\label{\detokenize{pattern:sequence-diagram}}
\noindent\sphinxincludegraphics{{ss}.png}

The use of the pattern can be achieved by defining a \sphinxstylestrong{concrete} Share class and \sphinxstylestrong{two abstract} classes Service and Feature.

So while the Share class can be created directly, the Share pattern specifies the behavior the pattern requires
basically the implementation of \sphinxcode{\sphinxupquote{function}}, \sphinxcode{\sphinxupquote{daemon}}, \sphinxcode{\sphinxupquote{pre}} and \sphinxcode{\sphinxupquote{post}} using unchanged the functionality of the
remaining operations related to the structure of the pattern.

In the pattern structure, the subscription to the Share service is not
places no constraints. So you can subscribe to multiple Share or only one.
Another aspect is the interaction between daemon and function. The code associated with function is received by
the get operator \sphinxstylestrong{immediately} after executed with daemon.

The interaction between \sphinxcode{\sphinxupquote{function}} and \sphinxcode{\sphinxupquote{daemon}} takes place through unforeseen protocols
from the pattern. For example, the protocol can be used as \sphinxstylestrong{0MQ}. There
correct interaction between function and daemon \sphinxstylestrong{is guaranteed by the fact that both they are developed by the same programmer}.

\begin{sphinxadmonition}{note}{Note:}
ZeroMQ is a high\sphinxhyphen{}performance asynchronous messaging library,
aimed at use in distributed or concurrent applications. It provides a \sphinxstylestrong{message queue}, but
unlike message\sphinxhyphen{}oriented middleware, a \sphinxstylestrong{ZeroMQ system can run without a dedicated message broker}.
\end{sphinxadmonition}

The attach operation is used to register with a \sphinxcode{\sphinxupquote{subscriber}}. Any positive can be a subscriber however it is plausible that only a few provide
this service. The search for a service is done through a descriptive string.
Could be an ontological descriptor or simply a protocol \sphinxstylestrong{MIB} string
SNMP. The match function was used in the specification and the pattern allows the nested call of services


\chapter{Documentation}
\label{\detokenize{documentation:documentation}}\label{\detokenize{documentation::doc}}\begin{quote}

\sphinxstyleemphasis{One documentation to rule them all,
one documentation to find them,
One documentation to bring them all
and in the darkness bind them.} %
\begin{footnote}[1]\sphinxAtStartFootnote
\sphinxstyleemphasis{Modified version of The One Ring, the central plot element in J. R. R. Tolkien’s The Lord of the Rings (1954\textendash{}55). It first appeared in the earlier story The Hobbit (1937) as a magic ring that grants the wearer invisibility}
%
\end{footnote}
\end{quote}

The entire generated documentation that is being used is the result of the union between \sphinxstylestrong{Sphinx} and \sphinxcode{\sphinxupquote{sphinx\sphinxhyphen{}lua}},
two tools dedicated to the generation of texts starting from mere and pure code.


\section{Tools with which it was made}
\label{\detokenize{documentation:tools-with-which-it-was-made}}
Usually the part of the code documentation in Lua is somewhat \sphinxstylestrong{boring} and \sphinxstylestrong{stylistically questionable}.

From this premise comes the integration of Sphinx %
\begin{footnote}[2]\sphinxAtStartFootnote
\sphinxstyleemphasis{Sphinx is a documentation generator written and used by the Python community. It is written in Python, and also used in other environments.}
%
\end{footnote} within the project which has made the entire software
park a \sphinxstyleemphasis{pleasant} and \sphinxstyleemphasis{clear} product in \sphinxstylestrong{understanding its API}.


\section{Integration with Lua}
\label{\detokenize{documentation:integration-with-lua}}
For the documentation of the classes created, therefore, we relied on a tool that puts \sphinxstylestrong{Python} \sphinxstyleemphasis{(given the nature of Sphinx)} and \sphinxstylestrong{Lua} in symbiosis.

This tool is called \sphinxcode{\sphinxupquote{sphinx\sphinxhyphen{}lua}} %
\begin{footnote}[3]\sphinxAtStartFootnote
\sphinxhref{https://github.com/boolangery/sphinx-lua}{GitHub page of the project sphinx\sphinxhyphen{}lua}
%
\end{footnote}.

It can be easily installed using the following command:

\begin{sphinxVerbatim}[commandchars=\\\{\}]
\PYG{n}{pip} \PYG{n}{install} \PYG{n}{sphinx}\PYG{o}{\PYGZhy{}}\PYG{n}{lua}
\end{sphinxVerbatim}

Once that you installed this on your machine your could simply start to document your class in this way:

\begin{sphinxVerbatim}[commandchars=\\\{\}]
\PYG{c+c1}{\PYGZhy{}\PYGZhy{}\PYGZhy{} Define a car.}
\PYG{c+c1}{\PYGZhy{}\PYGZhy{}\PYGZhy{} @class MyOrg.Car}
\PYG{k+kd}{local} \PYG{n}{cls} \PYG{o}{=} \PYG{n}{class}\PYG{p}{(}\PYG{p}{)}

\PYG{c+c1}{\PYGZhy{}\PYGZhy{}\PYGZhy{} @param foo number}
\PYG{k+kr}{function} \PYG{n+nc}{cls}\PYG{p}{:}\PYG{n+nf}{test}\PYG{p}{(}\PYG{n}{foo}\PYG{p}{)}
\PYG{k+kr}{end}
\end{sphinxVerbatim}


\chapter{Implementation}
\label{\detokenize{code:implementation}}\label{\detokenize{code::doc}}
The following page will show what has been \sphinxstylestrong{our implementation} of this pattern by providing the documentation of the APIs created.


\section{Share}
\label{\detokenize{code:share}}
The \sphinxstylestrong{Share} class is the beating heart of the pattern of the same name, the \sphinxstyleemphasis{raison d’etre} of the same and spokesperson for the
current of thought that characterized the project in its entirety: \sphinxstylestrong{elegance is everything}. The class has the list of services
that a device proudly makes available to all, and offers features to add or remove others. But the main responsibility of this
class is to perform the \sphinxcode{\sphinxupquote{discovery}} function, a symbol of an endless adventure, an adventure that begins in the search for the
services that best lend themselves to the arduous and meticulous work that the calling service requires.

The function uses the \sphinxstylestrong{MDNS protocol} to first calculate the ip address of each device on the network,
then subsequently examine the table of services available from the same.

\begin{sphinxadmonition}{note}{Note:}
In computer networking, the multicast DNS (mDNS) protocol resolves hostnames to IP addresses
within small networks that do not include a local name server. It is a zero\sphinxhyphen{}configuration service,
using essentially the same programming interfaces, packet formats and operating semantics as the
unicast Domain Name System (DNS). Although Stuart Cheshire designed mDNS as a stand\sphinxhyphen{}alone protocol,
it can work in concert with standard DNS servers.
\end{sphinxadmonition}

The \sphinxcode{\sphinxupquote{discovery}} function is called by \sphinxcode{\sphinxupquote{call}}, a function that finds an origin but perhaps does not find an end, a very long path that
only the best services can undertake to the end. And so it is that from these services the transfer of \sphinxstylestrong{knowledge} takes place between the
calling device and the called device, a knowledge obtained in a transversal, unorthodox way, based on the code and not on the network.
The caller does not get knowledge directly, if he has to conquer it by executing the code that was provided to him by the caller.

Share doesn’t give you fish, but it can help you fish. Elegance is everything.
\index{Share (built\sphinxhyphen{}in class)@\spxentry{Share}\spxextra{built\sphinxhyphen{}in class}}

\begin{fulllineitems}
\phantomsection\label{\detokenize{code:Share}}\pysigline{\sphinxbfcode{\sphinxupquote{class }}\sphinxbfcode{\sphinxupquote{Share}}}~\index{new() (Share method)@\spxentry{new()}\spxextra{Share method}}

\begin{fulllineitems}
\phantomsection\label{\detokenize{code:Share.new}}\pysiglinewithargsret{\sphinxbfcode{\sphinxupquote{ }}\sphinxbfcode{\sphinxupquote{new}}}{}{}
The constructor of the object Share
\begin{quote}\begin{description}
\item[{Returns}] \leavevmode
The new Share just created with the table of available services

\item[{Return type}] \leavevmode
{\hyperref[\detokenize{code:Share}]{\sphinxcrossref{Share}}}

\end{description}\end{quote}

\end{fulllineitems}

\index{attach() (Share method)@\spxentry{attach()}\spxextra{Share method}}

\begin{fulllineitems}
\phantomsection\label{\detokenize{code:Share.attach}}\pysiglinewithargsret{\sphinxbfcode{\sphinxupquote{ }}\sphinxbfcode{\sphinxupquote{attach}}}{\emph{s}}{}
This method inserts a service into the table of available services
\begin{quote}\begin{description}
\item[{Parameters}] \leavevmode
\sphinxstyleliteralstrong{\sphinxupquote{s}} ({\hyperref[\detokenize{code:Service}]{\sphinxcrossref{\sphinxstyleliteralemphasis{\sphinxupquote{Service}}}}}) \textendash{} The service to add

\end{description}\end{quote}

\end{fulllineitems}

\index{detach() (Share method)@\spxentry{detach()}\spxextra{Share method}}

\begin{fulllineitems}
\phantomsection\label{\detokenize{code:Share.detach}}\pysiglinewithargsret{\sphinxbfcode{\sphinxupquote{ }}\sphinxbfcode{\sphinxupquote{detach}}}{\emph{s}}{}
This method removes a service from the table of available services
\begin{quote}\begin{description}
\item[{Parameters}] \leavevmode
\sphinxstyleliteralstrong{\sphinxupquote{s}} ({\hyperref[\detokenize{code:Service}]{\sphinxcrossref{\sphinxstyleliteralemphasis{\sphinxupquote{Service}}}}}) \textendash{} The service to remove

\end{description}\end{quote}

\end{fulllineitems}

\index{is\_present() (Share method)@\spxentry{is\_present()}\spxextra{Share method}}

\begin{fulllineitems}
\phantomsection\label{\detokenize{code:Share.is_present}}\pysiglinewithargsret{\sphinxbfcode{\sphinxupquote{ }}\sphinxbfcode{\sphinxupquote{is\_present}}}{\emph{s}, \emph{t}}{}
This method search a service from the services table and returns true if it finds an occurrence
\begin{quote}\begin{description}
\item[{Parameters}] \leavevmode\begin{itemize}
\item {} 
\sphinxstyleliteralstrong{\sphinxupquote{s}} ({\hyperref[\detokenize{code:Service}]{\sphinxcrossref{\sphinxstyleliteralemphasis{\sphinxupquote{Service}}}}}) \textendash{} The service to search

\item {} 
\sphinxstyleliteralstrong{\sphinxupquote{t}} (\sphinxstyleliteralemphasis{\sphinxupquote{table}}) \textendash{} The table on which doing the search

\end{itemize}

\item[{Returns}] \leavevmode
True if the service is present, false otherwise

\item[{Return type}] \leavevmode
boolean

\end{description}\end{quote}

\end{fulllineitems}

\index{discovery() (Share method)@\spxentry{discovery()}\spxextra{Share method}}

\begin{fulllineitems}
\phantomsection\label{\detokenize{code:Share.discovery}}\pysiglinewithargsret{\sphinxbfcode{\sphinxupquote{ }}\sphinxbfcode{\sphinxupquote{discovery}}}{\emph{macro\_mib}}{}~\begin{quote}\begin{description}
\item[{Parameters}] \leavevmode
\sphinxstyleliteralstrong{\sphinxupquote{macro\_mib}} (\sphinxstyleliteralemphasis{\sphinxupquote{any}}) \textendash{} 

\end{description}\end{quote}

\end{fulllineitems}

\index{find() (Share method)@\spxentry{find()}\spxextra{Share method}}

\begin{fulllineitems}
\phantomsection\label{\detokenize{code:Share.find}}\pysiglinewithargsret{\sphinxbfcode{\sphinxupquote{ }}\sphinxbfcode{\sphinxupquote{find}}}{\emph{macro\_mib}}{}
Internal function that retrieve the set of services with the corresponding prefix
\begin{quote}\begin{description}
\item[{Parameters}] \leavevmode
\sphinxstyleliteralstrong{\sphinxupquote{macro\_mib}} (\sphinxstyleliteralemphasis{\sphinxupquote{str}}) \textendash{} The prefix of the mib to search

\item[{Returns}] \leavevmode
The set of corresponding services

\item[{Return type}] \leavevmode
table

\end{description}\end{quote}

\end{fulllineitems}

\index{open\_udp\_socket() (Share method)@\spxentry{open\_udp\_socket()}\spxextra{Share method}}

\begin{fulllineitems}
\phantomsection\label{\detokenize{code:Share.open_udp_socket}}\pysiglinewithargsret{\sphinxbfcode{\sphinxupquote{ }}\sphinxbfcode{\sphinxupquote{open\_udp\_socket}}}{\emph{ip}, \emph{macro\_mib}, \emph{result}}{}
Internal function used to establish a remote connection with udp socket
\begin{quote}\begin{description}
\item[{Parameters}] \leavevmode\begin{itemize}
\item {} 
\sphinxstyleliteralstrong{\sphinxupquote{ip}} (\sphinxstyleliteralemphasis{\sphinxupquote{str}}) \textendash{} The ip of the remote device

\item {} 
\sphinxstyleliteralstrong{\sphinxupquote{macro\_mib}} (\sphinxstyleliteralemphasis{\sphinxupquote{str}}) \textendash{} MIB of the service owned by the remote service

\item {} 
\sphinxstyleliteralstrong{\sphinxupquote{result}} (\sphinxstyleliteralemphasis{\sphinxupquote{table}}) \textendash{} The table used to save all results

\end{itemize}

\end{description}\end{quote}

\end{fulllineitems}


\end{fulllineitems}



\section{Service}
\label{\detokenize{code:service}}
The \sphinxstylestrong{Service} class represents any functionality made available by the device that owns it.

The responsibility of this class, in addition to providing the result of the computation with the parameters
requested by the caller through the \sphinxcode{\sphinxupquote{daemon}} function, is to establish a safe and reliable communication protocol with the same.

Each Service has a personal \sphinxstylestrong{MIB} which makes it \sphinxstylestrong{unique} in the environment in which it operates.
\index{Service (built\sphinxhyphen{}in class)@\spxentry{Service}\spxextra{built\sphinxhyphen{}in class}}

\begin{fulllineitems}
\phantomsection\label{\detokenize{code:Service}}\pysigline{\sphinxbfcode{\sphinxupquote{class }}\sphinxbfcode{\sphinxupquote{Service}}}~\index{new() (Service method)@\spxentry{new()}\spxextra{Service method}}

\begin{fulllineitems}
\phantomsection\label{\detokenize{code:Service.new}}\pysiglinewithargsret{\sphinxbfcode{\sphinxupquote{ }}\sphinxbfcode{\sphinxupquote{new}}}{\emph{i}, \emph{f}, \emph{d}, \emph{p}, \emph{...}}{}
The constructor of the object Service
\begin{quote}\begin{description}
\item[{Parameters}] \leavevmode\begin{itemize}
\item {} 
\sphinxstyleliteralstrong{\sphinxupquote{i}} (\sphinxstyleliteralemphasis{\sphinxupquote{str}}) \textendash{} The MIB of the current Service

\item {} 
\sphinxstyleliteralstrong{\sphinxupquote{f}} (\sphinxstyleliteralemphasis{\sphinxupquote{str}}) \textendash{} The function wrapped in a string that allows the communication with the inner protcol

\item {} 
\sphinxstyleliteralstrong{\sphinxupquote{d}} (\sphinxstyleliteralemphasis{\sphinxupquote{function}}) \textendash{} The defined daemon that share data with the function of the same Service

\item {} 
\sphinxstyleliteralstrong{\sphinxupquote{p}} (\sphinxstyleliteralemphasis{\sphinxupquote{function}}) \textendash{} The pre\sphinxhyphen{}condition necessary to checking

\item {} 
\sphinxstyleliteralstrong{\sphinxupquote{vararg}} (\sphinxstyleliteralemphasis{\sphinxupquote{any}}) \textendash{} The set of features

\end{itemize}

\item[{Returns}] \leavevmode
The new Service just created or nil in case of any issues

\item[{Return type}] \leavevmode
{\hyperref[\detokenize{code:Service}]{\sphinxcrossref{Service}}}

\end{description}\end{quote}

\end{fulllineitems}


\end{fulllineitems}



\section{Feature}
\label{\detokenize{code:feature}}
The \sphinxstylestrong{Feature} class represents the entity that contains the complete set of services that perform the same function.

In fact, this class has a \sphinxstylestrong{MIB} that identifies a macro category within its environment. Each Feature has a feature called \sphinxcode{\sphinxupquote{post}} which
checks the postconditions of each result received.

The responsibility of this class is also to identify the list of desired services that share the same network through the \sphinxcode{\sphinxupquote{call}} function.
\index{Feature (built\sphinxhyphen{}in class)@\spxentry{Feature}\spxextra{built\sphinxhyphen{}in class}}

\begin{fulllineitems}
\phantomsection\label{\detokenize{code:Feature}}\pysigline{\sphinxbfcode{\sphinxupquote{class }}\sphinxbfcode{\sphinxupquote{Feature}}}~\index{new() (Feature method)@\spxentry{new()}\spxextra{Feature method}}

\begin{fulllineitems}
\phantomsection\label{\detokenize{code:Feature.new}}\pysiglinewithargsret{\sphinxbfcode{\sphinxupquote{ }}\sphinxbfcode{\sphinxupquote{new}}}{\emph{i}, \emph{p}}{}
The constructor of the object Feature
\begin{quote}\begin{description}
\item[{Parameters}] \leavevmode\begin{itemize}
\item {} 
\sphinxstyleliteralstrong{\sphinxupquote{i}} (\sphinxstyleliteralemphasis{\sphinxupquote{str}}) \textendash{} The MIB of the current Feature

\item {} 
\sphinxstyleliteralstrong{\sphinxupquote{p}} (\sphinxstyleliteralemphasis{\sphinxupquote{function}}) \textendash{} The post\sphinxhyphen{}condition necessary to checking

\end{itemize}

\item[{Returns}] \leavevmode
The new Feature just created or nil in case of any issues

\item[{Return type}] \leavevmode
{\hyperref[\detokenize{code:Feature}]{\sphinxcrossref{Feature}}}

\end{description}\end{quote}

\end{fulllineitems}

\index{call() (Feature method)@\spxentry{call()}\spxextra{Feature method}}

\begin{fulllineitems}
\phantomsection\label{\detokenize{code:Feature.call}}\pysiglinewithargsret{\sphinxbfcode{\sphinxupquote{ }}\sphinxbfcode{\sphinxupquote{call}}}{\emph{...}}{}
A stub that searches, verifies, executes and produces the results related to a remote service
\begin{quote}\begin{description}
\item[{Parameters}] \leavevmode
\sphinxstyleliteralstrong{\sphinxupquote{vararg}} (\sphinxstyleliteralemphasis{\sphinxupquote{any}}) \textendash{} The parameters that are called are a regular expression and the parameters on which to perform the operation

\item[{Returns}] \leavevmode
Produces a boolean indicating whether the operation is successful and a table with the values ​​produced by the requested service

\item[{Return type}] \leavevmode
table or boolean

\end{description}\end{quote}

\end{fulllineitems}


\end{fulllineitems}



\chapter{Example}
\label{\detokenize{example:example}}\label{\detokenize{example::doc}}
In the following section some examples will be shown that have the purpose of making
the substance of the pattern \sphinxstylestrong{more concretely} understood and what some uses of this pattern may be.


\section{Abstract}
\label{\detokenize{example:abstract}}
The concept of service management is of significant importance, as is the dynamism of the
same, which plays an important role in the \sphinxstylestrong{reliability}, \sphinxstylestrong{stability} and \sphinxstylestrong{correctness} of the services provided.

In this context, \sphinxstylestrong{Share} allows you to change the semantics of a service according to the
devices available at a given moment, thus leaving \sphinxstylestrong{complete freedom} on the low\sphinxhyphen{}level
interaction model, preferring the modularity and composition of the services.

In order to correctly verify the constraints defined by the communicating devices,
we have relied on an approach based on the design \sphinxstyleemphasis{by contract}, which defines \sphinxcode{\sphinxupquote{pre\sphinxhyphen{}conditions}} and
post\sphinxhyphen{}conditions on the values ​​provided respectively by the caller and the called party. In this
way you have a guarantee on the validity and consistency of the data obtained.


\section{Calculation of the square root}
\label{\detokenize{example:calculation-of-the-square-root}}
For simplicity, assume that the service requested is the mere calculation of the
square root on a set parameter. First of all, the device requesting the service must know the
\sphinxcode{\sphinxupquote{MIB}}, that is a regular expression that uniquely identifies a class of semantically equivalent functionality.
\begin{description}
\item[{First step}] \leavevmode
For example, if the category \sphinxstylestrong{“Mathematics”} had the number \sphinxcode{\sphinxupquote{1}}
as specific \sphinxcode{\sphinxupquote{MIB}} and the square root had the number \sphinxcode{\sphinxupquote{2}},
the mib on which to make the call would be \sphinxcode{\sphinxupquote{1.2.*}}.
In this way the device is able to search, and eventually
\sphinxstylestrong{discover}, all the devices that provide services that at that
moment calculate the square root.
\begin{description}
\item[{Client side}] \leavevmode
\begin{sphinxVerbatim}[commandchars=\\\{\}]
\PYG{c+c1}{\PYGZhy{}\PYGZhy{} 2 is the number on which we calculate the square root}
\PYG{n}{services}\PYG{p}{[}\PYG{l+s+s2}{\PYGZdq{}}\PYG{l+s+s2}{1.2.1.0}\PYG{l+s+s2}{\PYGZdq{}}\PYG{p}{]}\PYG{p}{.}\PYG{n}{features}\PYG{p}{[}\PYG{l+m+mi}{1}\PYG{p}{]}\PYG{p}{:}\PYG{n}{call}\PYG{p}{(}\PYG{l+m+mi}{2}\PYG{p}{)}
\end{sphinxVerbatim}

\item[{Server side}] \leavevmode
\begin{sphinxVerbatim}[commandchars=\\\{\}]
\PYG{n}{udp\PYGZus{}discovery}\PYG{p}{:}\PYG{n}{sendto}\PYG{p}{(}\PYG{n}{Utilities}\PYG{p}{:}\PYG{n}{table\PYGZus{}to\PYGZus{}string}\PYG{p}{(}\PYG{n}{disc}\PYG{p}{:}\PYG{n}{find}\PYG{p}{(}\PYG{n}{data\PYGZus{}discovery}\PYG{p}{)}\PYG{p}{)}\PYG{p}{,} \PYG{n}{ip\PYGZus{}discovery}\PYG{p}{,} \PYG{n}{port\PYGZus{}discovery}\PYG{p}{)}
\end{sphinxVerbatim}

\end{description}

\item[{Second step}] \leavevmode
Once the table of available services has been obtained, the
calling device interrogates these services \sphinxstylestrong{one by one} by
sending them the parameter on which to perform the calculation
\sphinxstyleemphasis{(in this case the pre\sphinxhyphen{}conditions must verify that this parameter
is greater than zero)}.
\begin{description}
\item[{Client side}] \leavevmode
\begin{sphinxVerbatim}[commandchars=\\\{\}]
\PYG{n}{check\PYGZus{}param}\PYG{p}{(}\PYG{n}{mib}\PYG{p}{,} \PYG{p}{...}\PYG{p}{,} \PYG{n}{udp\PYGZus{}feature}\PYG{p}{)}
\end{sphinxVerbatim}

\item[{Server side}] \leavevmode
\begin{sphinxVerbatim}[commandchars=\\\{\}]
\PYG{k+kr}{if} \PYG{p}{(}\PYG{n}{services}\PYG{p}{[}\PYG{n}{mib}\PYG{p}{]}\PYG{p}{.}\PYG{n}{pre}\PYG{p}{(}\PYG{n}{param}\PYG{p}{)}\PYG{p}{)} \PYG{k+kr}{then}
    \PYG{n}{udp\PYGZus{}call}\PYG{p}{:}\PYG{n}{sendto}\PYG{p}{(}\PYG{n}{services}\PYG{p}{[}\PYG{n}{mib}\PYG{p}{]}\PYG{p}{.}\PYG{n}{func}\PYG{p}{,} \PYG{n}{ip\PYGZus{}call}\PYG{p}{,} \PYG{n}{port\PYGZus{}call}\PYG{p}{)}
\PYG{k+kr}{end}
\end{sphinxVerbatim}

\end{description}

\item[{Third step}] \leavevmode
In this phase, a \sphinxstylestrong{unicast communication} is initiated between the
devices (managed by the function and daemon functions) \sphinxstylestrong{which
guarantees confidentiality between the parties}, and in which the
calling device receives the function to be performed locally to
obtain the desired result.

\begin{sphinxadmonition}{note}{Note:}
In computer networking, \sphinxstylestrong{unicast} refers to a one\sphinxhyphen{}to\sphinxhyphen{}one
transmission from one point in the network to another point;
that is, \sphinxstylestrong{one sender and one receiver}, each identified by a
network address.
\end{sphinxadmonition}
\begin{description}
\item[{Client side}] \leavevmode
\begin{sphinxVerbatim}[commandchars=\\\{\}]
\PYG{p}{.}
\PYG{p}{.}
\PYG{n}{tcp}\PYG{p}{:}\PYG{n}{connect}\PYG{p}{(}\PYG{n}{host}\PYG{p}{,} \PYG{n}{port}\PYG{p}{)}\PYG{p}{;}
\PYG{n}{tcp}\PYG{p}{:}\PYG{n}{send}\PYG{p}{(}\PYG{n}{data}\PYG{o}{..}\PYG{l+s+s2}{\PYGZdq{}}\PYG{l+s+se}{\PYGZbs{}n}\PYG{l+s+s2}{\PYGZdq{}}\PYG{p}{)}\PYG{p}{;}
\PYG{p}{.}
\PYG{p}{.}
\PYG{k+kd}{local} \PYG{n}{s}\PYG{p}{,} \PYG{n}{status}\PYG{p}{,} \PYG{n}{partial} \PYG{o}{=} \PYG{n}{tcp}\PYG{p}{:}\PYG{n}{receive}\PYG{p}{(}\PYG{p}{)}
\end{sphinxVerbatim}

\item[{Server side}] \leavevmode
\begin{sphinxVerbatim}[commandchars=\\\{\}]
\PYG{n}{services}\PYG{p}{[}\PYG{n}{mib}\PYG{p}{]}\PYG{p}{.}\PYG{n}{daemon}\PYG{p}{(}\PYG{p}{)}
\end{sphinxVerbatim}

\end{description}

\item[{Fourth step}] \leavevmode
Once the result is obtained, the calling device can decide
whether to validate this result \sphinxstyleemphasis{(checking it in the post\sphinxhyphen{}conditions)}
or whether to move on to the next service. As soon as one of these
services is able to meet the established \sphinxcode{\sphinxupquote{post\sphinxhyphen{}conditions}}, the
\sphinxstylestrong{workflow} will end.
\begin{description}
\item[{Client side}] \leavevmode
\begin{sphinxVerbatim}[commandchars=\\\{\}]
\PYG{k+kr}{if} \PYG{p}{(}\PYG{n}{res} \PYG{o+ow}{and} \PYG{n}{self}\PYG{p}{.}\PYG{n}{post}\PYG{p}{(}\PYG{p}{...}\PYG{p}{,} \PYG{n}{res}\PYG{p}{)}\PYG{p}{)} \PYG{k+kr}{then}
    \PYG{n}{log}\PYG{p}{.}\PYG{n}{info}\PYG{p}{(}\PYG{l+s+s2}{\PYGZdq{}}\PYG{l+s+s2}{[POST\PYGZhy{}CONDITION SUCCESSFUL]}\PYG{l+s+s2}{\PYGZdq{}}\PYG{p}{)}
    \PYG{k+kr}{return} \PYG{n}{res}\PYG{p}{,} \PYG{k+kc}{true}
\PYG{k+kr}{end}
\end{sphinxVerbatim}

\end{description}

\end{description}


\section{Temperature measurement}
\label{\detokenize{example:temperature-measurement}}
The pattern also provides for the possibility of requesting services that
\sphinxstylestrong{do not provide parameters} from the calling device. For example, we
admit that a device needs to know the atmospheric room temperature to
perform a certain task. This request does not include any parameter
as the calculation of the temperature is a procedure that \sphinxstylestrong{does not involve
any operation on the data}, but simply makes a measurement and query a
possible thermometer.
\begin{description}
\item[{First step}] \leavevmode
The calling device will therefore simply need to invoke the \sphinxcode{\sphinxupquote{MIB}}
that identifies any service that has a thermometer, in this case
for purely demonstrative purposes it is assumed to be \sphinxcode{\sphinxupquote{2.1.*}}
\begin{description}
\item[{Client side}] \leavevmode
\begin{sphinxVerbatim}[commandchars=\\\{\}]
\PYG{c+c1}{\PYGZhy{}\PYGZhy{} the call function has no parameter as you can see}
\PYG{n}{services}\PYG{p}{[}\PYG{l+s+s2}{\PYGZdq{}}\PYG{l+s+s2}{2.1.1.0}\PYG{l+s+s2}{\PYGZdq{}}\PYG{p}{]}\PYG{p}{.}\PYG{n}{features}\PYG{p}{[}\PYG{l+m+mi}{1}\PYG{p}{]}\PYG{p}{:}\PYG{n}{call}\PYG{p}{(}\PYG{p}{)}
\end{sphinxVerbatim}

\item[{Server side}] \leavevmode
\begin{sphinxVerbatim}[commandchars=\\\{\}]
\PYG{n}{udp\PYGZus{}discovery}\PYG{p}{:}\PYG{n}{sendto}\PYG{p}{(}\PYG{n}{Utilities}\PYG{p}{:}\PYG{n}{table\PYGZus{}to\PYGZus{}string}\PYG{p}{(}\PYG{n}{disc}\PYG{p}{:}\PYG{n}{find}\PYG{p}{(}\PYG{n}{data\PYGZus{}discovery}\PYG{p}{)}\PYG{p}{)}\PYG{p}{,} \PYG{n}{ip\PYGZus{}discovery}\PYG{p}{,} \PYG{n}{port\PYGZus{}discovery}\PYG{p}{)}
\end{sphinxVerbatim}

\end{description}

\item[{Second step}] \leavevmode
Once the table of available services is obtained, the calling device
\sphinxstylestrong{interrogates} these services one by one (in this case the \sphinxcode{\sphinxupquote{pre\sphinxhyphen{}conditions}}
are always exceeded as no parameters are received).
\begin{description}
\item[{Client side}] \leavevmode
\begin{sphinxVerbatim}[commandchars=\\\{\}]
\PYG{n}{check\PYGZus{}param}\PYG{p}{(}\PYG{n}{mib}\PYG{p}{,} \PYG{p}{...}\PYG{p}{,} \PYG{n}{udp\PYGZus{}feature}\PYG{p}{)}
\end{sphinxVerbatim}

\item[{Server side}] \leavevmode
\begin{sphinxVerbatim}[commandchars=\\\{\}]
\PYG{k+kr}{if} \PYG{p}{(}\PYG{n}{services}\PYG{p}{[}\PYG{n}{mib}\PYG{p}{]}\PYG{p}{.}\PYG{n}{pre}\PYG{p}{(}\PYG{n}{param}\PYG{p}{)}\PYG{p}{)} \PYG{k+kr}{then}
    \PYG{n}{udp\PYGZus{}call}\PYG{p}{:}\PYG{n}{sendto}\PYG{p}{(}\PYG{n}{services}\PYG{p}{[}\PYG{n}{mib}\PYG{p}{]}\PYG{p}{.}\PYG{n}{func}\PYG{p}{,} \PYG{n}{ip\PYGZus{}call}\PYG{p}{,} \PYG{n}{port\PYGZus{}call}\PYG{p}{)}
\PYG{k+kr}{end}
\end{sphinxVerbatim}

\end{description}

\end{description}

The third and fourth steps correspond exactly to the example shown above.


\section{Nested call}
\label{\detokenize{example:nested-call}}
A \sphinxcode{\sphinxupquote{nested call}} is defined as any context in which the device called, \sphinxstylestrong{to fulfill the
result of the local computation}, it needs to in turn make a call to another service.


\subsection{In\sphinxhyphen{}depth explanation}
\label{\detokenize{example:in-depth-explanation}}
Home automation devices, smart industry, smart city, smart energy or any other type of device
they are installed over time and vary in number and type \sphinxstylestrong{without} a pre\sphinxhyphen{}arranged installation plan.

Providing services that include \sphinxstylestrong{their collaboration} is difficult because interconnection
and service interaction protocols are different and could vary over time.

In this context, the adaptive and very flexible nature of the pattern allows the
possibility \sphinxstylestrong{to perform nested calls between devices}, so as to further enhance the concept
of interoperability between the themselves and take full advantage of the
dynamism that distinguishes them.


\subsection{Concrete demonstration}
\label{\detokenize{example:concrete-demonstration}}
Let’s consider a sporting event, where racing bikes compete. Suppose it
is necessary calculate the speed of the bikes and compare them to
each other to understand which is the \sphinxstylestrong{best} rider.

The devices available are:
\begin{itemize}
\item {} 
A \sphinxstylestrong{big screen}, which shows a general ranking to the fans based on the average speed of the drivers

\item {} 
\sphinxstylestrong{Speed detectors} scattered across the track, which could be of different brands and calculate speeds with different measurement systems \sphinxstyleemphasis{(km/h, m/s, mph)}

\item {} 
A \sphinxstylestrong{classification device}, which sorts the set of all speeds detected in descending order and ensures uniformity between data \sphinxstyleemphasis{(converts m/s to km/ h or vice versa)}

\end{itemize}

In this context, the maxi\sphinxhyphen{}screen, in order to update the
data shown in the table, \sphinxstylestrong{makes a call to the classification device,
which in turn makes a call to the speed detectors to first receive
any updates on the drivers’ speeds}.

All these logics can be developed on the individual devices that have
competence on the actions they undertake according to the different
configurations that are detected allowing \sphinxcode{\sphinxupquote{flexibility}}, \sphinxcode{\sphinxupquote{adaptation}}
and \sphinxcode{\sphinxupquote{fault tolerance}} to the system.


\chapter{Future works}
\label{\detokenize{future:future-works}}\label{\detokenize{future::doc}}
In the near future it would certainly be interesting to implement
features like these:
\begin{itemize}
\item {} 
Creation of a domain language for \sphinxstylestrong{IoT} systems

\item {} 
Static analysis of the system performances

\item {} 
Integration with data cloud

\end{itemize}



\renewcommand{\indexname}{Index}
\printindex
\end{document}